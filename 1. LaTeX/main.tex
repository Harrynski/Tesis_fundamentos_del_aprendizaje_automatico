\documentclass{article}
\usepackage[spanish]{babel}
\usepackage[utf8]{inputenc} % Ensures proper handling of special characters
\usepackage{amsthm}
\usepackage{amsmath}
\usepackage{amsfonts}

% Define the unnumbered theorem
\newtheorem*{teorema}{Teorema}
%\newtheorem*{def}{Definicion}

\begin{document}


\title{Teoría del Aprendizaje Estadístico}
\author{Nicolas Silva Nash \\ Departamento de Matemática \\ Universidad Nacional del Comahue}
%\date{\today}
\maketitle

\section{Introducción}
La Teoría del Aprendizaje Estadístico proporciona la base teórica para muchos de los algoritmos de aprendizaje automático actuales y,
sin lugar a dudas, es una de las ramas más bellamente desarrolladas de la inteligencia artificial en general. Nació con el perceptrón de Rossenblat
y la escuela matemática de la Unión Soviética en la década de 1960, y ganó amplia popularidad en la década de 1990 tras el desarrollo de las llamadas 
Máquinas de Vectores de Soporte (\textit{SVM}, por sus siglas en inglés), que se han convertido en una herramienta estándar para el reconocimiento de 
patrones en muchas disciplinas, que van desde la visión por computadora hasta la biología computacional.\\

Proporcionar la base para nuevos algoritmos de aprendizaje no ha sido la única motivación para desarrollar la Teoría del Aprendizaje 
Estadístico. También ha sido una gesta de caracter filosófico, en el intento de responder a la pregunta de qué nos permite extraer conclusiones válidas 
a partir de datos empíricos.  

\section{El aprendizaje}

En este contexto, el \textit{aprendizaje} es el proceso a través del cual pueden inferirse reglas generales a partir de ejemplos. Nos interesa 
entender como una máquina -una computadora- puede resolver ciertos problemas sin conocer las reglas de antemano, solo a partir de ejemplos y a través
de un \textit{algoritmo de aprendizaje}. El objetivo es que la máquina pueda no solo aprender a reconocer las reglas que rigen a los ejemplos dados,
si no que también pueden generalizar dichas reglas para ejemplos que le serán presentados con posterioridad.\\

Llamamos a esta disciplina \textit{aprendizaje automático} (en inglés, \textit{Machine Learning}, literalmente ``aprendizaje de máquinas") y reconocemos
sus raíces en otras disciplinas: Estadística Matemática, Ciencias de la Computación e Inteligencia Artificial. Si bien el aprendizaje suele ser una
parte fundamental de la mayoría de los esfuerzos en materia de Inteligencia Artificial, el objetivo del Machine Learning es más acotado que el de su 
rama madre: En vez de intentar definir, explicar o generar comportamiento inteligente o \textit{inteligencia}, aquí nos interesa solamente descubrir los mecanismos
a través de los cuales las computadoras pueden resolver algunas tareas acotadas y bien definidas, y que en general escapan a soluciones que pueden
ser especificadas con una cantidad finita de código de programación (reglas determinísticas). \\

Con fines ilustrativos, nos centraremos primero en el más conocido de los problemas del Machine Learning, el de clasificación. Consideremos dos espacios
de variables: \(X\), llamado \textit{espacio de entrada}, e \(Y\), el  \textit{espacio de etiquetas}. En un problema de clasificación, deseamos poder etiquetar
correctamente elementos de \(X\) con los valores de \(Y\). Por ejemplo, podríamos querer clasificar un conjunto de datos, en alguna representación fija, de distintos 
objetos en una cantidad de etiquetas como: silla, cama, microondas, perro, gato. Si esta representación es en imágenes de \(N\times M\) pixeles en blanco y negro (realmente, 
matrices de orden \(N\times M\) con coeficientes reales en el interavalo \([0,1]\) que representan la intensidad de cada pixel, del negro al blanco), el espacio \(X\) es 
el conjunto de dichas matrices y el espacio \(Y\) las categorías distintas que corresponden a lo que las imágenes muestran. Con el fin de aprender, a un algoritmo se le 
muestran ejemplos de imágenes y sus respectivas etiquetas \((X_1,Y_1)\), \((X_2,Y_2)\), ..., \((X_n,Y_n)\), a partir de los cuales este debe encontrar una función
\(f: X\rightarrow Y\), que comete la menor cantidad de errores posibles. A esta función \(f\) la llamamos \textit{clasificador}.

\section{La historia del aprendizaje automático}

El primer modelo de aprendizaje automático, según Vapnik en [2], fue sugerido por F. Rosenblatt, un psicólogo estadounidense, al que llamó \textit{perceptrón}, y su introducción
constituye el comienzo del análisis matemático del aprendizaje. Conceptualmente, la idea del perceptrón no era nueva, estando presente en la litetura de Neurofisiología durante
varios años. Rosenblatt, sin embargo, se aventuró en describir el modelo como un programa para computadoras y demostró con simples experimentos que dicho modelo era generalizable.
El perceptrón fue construído como una solución a un problema particular dentro del aprendizaje automático, el reconocimiento de patrones. En el caso más sencillo, este problema
consiste en hallar una regla para separar datos en dos categorías distintas a partir de ejemplos.\\

Para construir la regla de separación, el perceptrón sigue el modelo más sencillo de neurona, propuesto previamente por McCulloch y Pitts, de acuerdo al cual una neurona recibe
\(n\) valores (o \textit{inputs}) en la forma de un vector \(x = (x^1,\dots, x^n) \in X \subset \mathbb{R}^n \) y genera una etiqueta (\textit{output}) 
\(y\in\{-1,+1\}\) a través de una dependecia funcional dada por
\[
y = sgn\{(w\cdot x)-b\}
\]

con \(\cdot\) el producto interno de vectores en \(\mathbb{R}^n\), \(b\) un valor de límite y un vector \(w\) que se genera en el proceso de aprendizaje. Geométricamente, una 
neurona divide el espacio \(X\) en dos regiones: en una la etiqueta \(y\) vale \(+1\) y en la otra \(-1\). Las dos regiones son separadas por el hiperplano:
\[
(w\cdot x) - b = 0
\]


El vector \(w\) y el escalar \(b\) determinan la posición del hiperplano y sus valores son aprendidos por el perceptrón. Cuando combinamos varias neuronas, el perceptrón
separa el espacio de entrada en en dos regiones lineales a trozos y no necesariamente conexas. En 1960 no era claro como elegir todos los parametros \((w_1,\dots,w_k)\) y
\((b_1,\cdots, b_k)\) de todas las neuronas, por lo que se fijaban los valores de las primeras \(k-1\) y se intentaba encontrar los valores deseables para la última
de ellas. Geométricamente, se transformaba el espacio de entrada $X$ en un nuevo espacio $Z$ (eligiendo coeficientes apropiados para las primeras $k-1$ neuronas) y luego
se utilizaban los datos de entrenamiento para construir un hiperplano que separe el plano $Z$.
Tomando prestados de la fisiología los conceptos de aprendizaje con estímulos de premios y castigos, Rosenblat propuso un simple algoritmo para hallar estos 
coeficientes de manera iterativa, el cual describiremos a continuación.

DESCRIBIR perceptrón.\\

En 1962 Novikoff demostró el primer teorema relacionado al perceptrón. Podemos decir que este teorema inció propiamente la teoría del aprendizaje.

\begin{teorema}
Dado un conjunto de datos de entrenamiento como el descrito previamente, de manera que\\
1) La norma de los vectores de entrenamiento $z_i$ está acotada por una constante $R$:
$$
|z_i| \leq R, \qquad \forall i=1,2,\dots,k
$$
2) Los datos de entrenamiento pueden separados con un margen $\rho$:
$$
\sup_{w} \min_{i} y_i(z_i\cdot w) > \rho
$$
3) Los datos son alimentandos al perceptrón una cantidad \textit{suficiente} de veces.\\

Entonces el algoritmo encuentra el hiperplano que separa los datos de entrenamiento, luego de a lo
sumo $N$ correcciones, con $N$ verificando:
$$
N \leq \left [ \frac{R^2}{\rho^2}\right ]
$$

\end{teorema}

Resaltamos este teorema porque jugó un papel fundamental en la creación de la teoría del aprendizaje, conectando el
principio de minimización de errores en el conjunto de datos de entrenamiento con la capacidad de generalización
de los algoritmos de clasificación y su causa.

\section{Hacia la formalización}

Volviendo al caso de clasificación binaria en aprendizaje supervisado, partimos de ejemplos (datos de entrenamiento) 
en un espacio de entrada $X$ con alguna de las dos posibles etiquetas del espacio $Y=\{-1,+1\}$. Aquí, \textit{aprender}
se reduce a estimar una relación funcional $f:X\rightarrow Y$, el clasificador. Un algoritmo de aprendizaje es aquel
que a partir de los datos de entrenamiento construye una función $f$. Nos interesa construir una teoría que no asuma
de manera estricta nada acerca de $X$ e $Y$, pero nos permitimos asumir ciertas cosas del mecanismo que genera los
datos de entrenamiento. En particular, asumiremos que existe una \textit{distribución de probabilidad conjunta} $P=P(X,Y)$
sobre $X\times Y$ y que las muestras son tomadas de forma independiente de esta distribución de forma \textit{iid} -independiente
e idénticamente distribuída-. Notemos lo siguiente:
\begin{enumerate}
    \item \textit{No imponemos condiciones a la distribución de probabilidad $P$}. La gran diferencia que encontramos entre la Estadística
    tradicional y la teoría del aprendizaje estadístico es que en esta última trabajamos de manera agnóstica a la distribución
    que genera las muestras y deseamos llegar a conclusiones generales.
    \item \textit{Consideramos a las etiquetas de manera no determinística.} Consideramos a $P$ como una distribución de probabilidad
    no solo sobre las instancias de $X$, si no también sobre las propias etiquetas de $Y$. Por lo tanto, estas últimas no son solo 
    funciones tradicionales de los datos en $X$, si no que ellas mismas pueden ser aleatorias. Tenemos al menos dos buenas razones
    para tomar esta consideración: por un lado, el proceso de generación de datos puede tener ruido al asignar etiquetas (por ejemplo
    tomemos el caso de un detector de spam basado en la opinión de etiquetadores humanos que clasifican emails con un porcentaje de error; incluso
    los humanos pueden clasificar incorrectament algunos de esos emails), y por otro, ciertos problemas se prestan a que existan
    clases que se solapan (pensemos en la dificultad en diferenciar a un perro de un gato en una fotografías que los captura desde
    una gran distancia o con baja resolución).\\
    En la práctica, en vez de asignar etiquetas a los elementos en $X$ de manera determinística, daremos la probabilidad condicional
    de la etiqueta $y$ dado el valor $x$. En el caso de clasificación binaria, basta solo dar la probabilidad $P(Y=1|X=x)$ de que la etiqueta
    tenga valor $Y=1$, dado que la restante es complementaria:
    $$
    P(Y=-1|X=x) \quad = \quad 1 - P(Y=1|X=x)
    $$
    Ciertos problemas que hagan uso de datos con etiquetas con poco ruido nos llevaran naturalmente a probabilidades condicionales cercanas
    a $0$ y $1$, dejando un margen de error pequeño, pero cuando tratemos con solapamiento de clases, las probabilidades condicionales
    pueden acercarse a $\frac{1}{2}$ para cada etiqueta. Independientemente de la causa, que las probabilidades condicionales sobre
    las etiquetas se acerquen a $\frac{1}{2}$ vuelve más dificil el aprendizaje, dado que crece el número de errores del clasificador.
    \item \textit{Muestro independiente}. Una de las condiciones más fuertes que imponemos en la teoría del aprendizaje estadístico es
    que asumimos que las muestras son tomadas de forma independiente. En muchas aplicaciones, esta suposición está justificada, pero
    hay ramas muy importantes de la disciplina en donde esto no se cumple, por ejemplo en el análisis de series de tiempo, en donde
    la secuencialidad de los datos viola la condición de iid (cada valor depende en alguna medida de los anteriores). Esto es también 
    cierto para las aplicaciones a lenguaje natural, y constituye una de las razones principales por las cuales esta rama más moderna 
    del aprendizaje tiene bases teóricas menos fuertes que el aprendizaje automático tradicional.
    \item \textit{La distribución P es fija}. Al no considerar al tiempo como un parámetro, ni existir un orden en las muestras, asumimos
    que la distribución que las origina es siempre la misma. Esto, como en el punto anterior, no se cumple en las series de tiempo. Otro
    caso donde se viola esta suposición es en aquellos problemas en donde la distribución de probabilidad de los datos de entrenamiento no
    coincide con el de los datos posteriores, llamado \textit{covariate shift}, por ejemplo en un sistema de \textit{scoring} de usuarios
    de una empresa que crece súbitamente y que suma a personas que no se corresponden a los perfiles que existían originalmente en su base
    de datos (e.g. se admite que inmigrantes no bancarizados y sobre los que no hay datos previos accedan a préstamos).
    \item \textit{La distribución P es desconocida al momento de aprender}. Si conociéramos de antemano la probabilidad condicional $P$,
    el problema del aprendizaje sería trivial pues podríamos siempre determinar el mejor clasificador posible (aunque no sea perfecto,
    dada la naturaleza aleatoria de las etiquetas). Solo tenemos acceso a $P$ de manera indirecta, a través de las muestras. Intuitivamente,
    esto nos hace pensar que, consiguiendo un número lo suficientemente grande de muestras, podemos aproximar las propiedades de la distribución
    $P$, pero con errores. Uno de los principales logros de la teoría del aprendizaje estadístico es brindarnos un marco teórico para
    acotar este error.
\end{enumerate}

Para saber qué tan bien se comporta un clasificador $f$, necesitamos medir sus equivocaciones. Para esto, definiremos una \textit{función de pérdida},
$\ell$, que le asigne un valor al hecho de que $f$ clasifique a cierto $x\in X$ con la etiqueta $y\in Y$. Llamermos \textit{costo} a dicho valor, 
dado que más adelante penalizaremos al clasificador en base a los errores que cometa a través del alogritmo de aprendizaje.\\
El ejemplo más sencillo de función de pérdida es la ``pérdida-0-1'', que le asigna un costo de $0$ a una instancia de clasificación correcta y un costo
$1$ a una incorrecta, es decir:
$$
\ell(X,Y,f(X)):= \begin{cases}
    1 \qquad \text{si } f(X) \neq Y \\
    0 \qquad \text{si } f(X) = Y
\end{cases}
$$
En problemas de regresión, la función de pérdida más conocida es el \textit{error cuadrático medio}, dado por
$$
\ell(X,Y,f(X)):= (Y-f(X))^2
$$
Por convención, una pérdida igual a $0$ implica una clasificación perfecta y valores mayores implican peor clasificación. Es decir,
el aprendizaje suele implicar un desafío de optimización en dónde deseamos hallar el mínimo de la función de perdida.

Mientras que la función de pérdida mide el error del clasificador en un punto individual $x\in X$, llamamos \textit{riesgo}, $\mathcal{R}$, 
del clasificador a la pérdida esperada sobre todos los datos generados por la distribución de probabilidad $P$. Es decir
$$
\mathcal{R}(f) := E\left( \ell(X,Y,f(X)) \right)
$$
Desde luego que otra función $g$ es un mejor clasificador que $f$ para un problema dado si su riesgo es más bajo, es decir si $\mathcal{R}(g)
< \mathcal{R}(f)$, por lo que el mejor clasificador de todos es aquel con el riesgo más bajo.\\

Algo que no hemos considerado aún es si los clasificadores $f$ tienen alguna característica especial. Para formalizarlo, tomaremos funciones de un
espacio de funciones $\mathcal{F}$ que aplican $X$ en $Y$. En un principio, parecería aceptable tomar el espacio de todas las funciones posibles, o 
más precisamente, el conjunto de todas las funciones \textit{medibles} que aplican $X$ en $Y$, $\mathcal{F}_{all} = \{f \text{ medibles  } | 
\; f: X \rightarrow Y \}$. En este caso, podemos señalar cuál es el clasificador ideal, dada la distribución $P$, al que llamamos
\textit{clasificador Bayesiano}, $f_{\text{Bayes}}$, y al que definimos como:
$$
f_{\text{Bayes}} := \begin{cases}
1 \qquad \; \; \text{  si } P(Y=1 | X=x) \geq \frac{1}{2}\\
-1 \qquad \text{en otro caso }
\end{cases}
$$

Observermos que, en caso de que las etiquetas fueran determinísticas, es decir donde $P(Y=y|X=x)=1$ para cada $x\in X$ con su
respectiva etiqueta $y$, $f_{\text{Bayes}}$ eligiría correctamente en todos los casos.\\
De existir un ligero solapamiento de clases de tal manera que para un cierto $x$ tengamos $P(Y=1|X=x) = 0.9$, entonces tendríamos
que en la mayoría de los casos la etiqueta de $x$ es $+1$ y, siendo este el valor elegido por $f_{\text{Bayes}}$, el clasificador
Bayesiano estaría en lo correcto.\\

En la práctica no es posible computar directamente el clasificador Bayesiano dado que, como dijimos, la distribución de probabilidad
conjunta $P$ es desconocida. Sin embargo, este clasificador es una herramienta teórica que nos permite formular el problema
estandar de la clasificación binaria, el cual es:\\

\textit{Dado un conjunto de datos de entrenamiento $ \{ (X_1,Y_1),\dots,(X_n, Y_m)\}$ obtenidos iid de una distribución $P$, y dada una
función de pérdida $\ell$, deseamos construir una funión clasificadora $f:X\rightarrow Y$ cuyo riesgo $\mathcal{R}(f)$ sea lo
más cercano posible al riesgo de $f_{\text{Bayes}}$.}\\

Notemos que no solo es imposible to computar el error del clasificador Bayesiano, sino también el propio riesgo de cualquier clasificador
$f$. Es decir, dado un problema definido (minimizar el riesgo del clasificador), con una solución ideal que podemos escribir (el propio
clasificador Bayesiano), no tenemos manera de computar ninguna cosa de utilidad. Aquí es donde la teoría de aprendizaje estadístico
nos permite llegar a resultados y obtener garantías de la utilidad de esas soluciones.






\bigbreak
\pagebreak
\section{Bibliografía}

[1] \textit{Statistical Learning Theory: Models, Concepts and Results} - von Luxburg, Schölkopf (2008)

[2] \textit{The Nature of Statistical Learning Theory, second edition} - Vladimir Vapnik (2000)





\end{document}